\documentclass{jamk-thesis}

\title{Example title}
\undertitle{Under title}
\tyyppi{test thesis}
\author{Firstname Lastname}
\date{Month Year}
\ala{Information and communication technology}
\tutkinto{Degree Programme in Software Engineering}


\begin{document}
\maketitle
\newpage

% Keskeneräinen kuvaus
% TODO: Siirrä cls tiedostoon, \makedescription komennon sisälle
\pagenumbering{gobble}
\clearpage
\begin{table}
\small
\begin{tabular}{|l|l|l|}
\hline
\multirow{3}{*}{
\begin{tabular}[c]{@{}l@{}}
    Tekijä(t)\\
\end{tabular}} &
\begin{tabular}[c]{@{}l@{}}
    Julkaisun laji\\
    Opinnäytetyö, AMK\end{tabular} &
\begin{tabular}[c]{@{}l@{}}
    Päivämäärä\\
    Toukokuu 2019
    %\mydate
\end{tabular}
\\ \cline{2-3} &
\multirow{2}{*}{
\begin{tabular}[c]{@{}l@{}}
    Sivumäärä\\
    \pageref{LastPage}
\end{tabular}}   &
\begin{tabular}[c]{@{}l@{}}
    Julkaisun kieli:\\
    Englanti\end{tabular}
\\ \cline{3-3} & &
    Verkkojulkaisulupa: x
\\ \hline
\multicolumn{3}{|l|}{
\begin{tabular}[c]{@{}l@{}}
    Työn nimi\\
    \textbf{Nimi}
    \\ \\
\end{tabular}}
\\ \hline
\multicolumn{3}{|l|}{
\begin{tabular}[c]{@{}l@{}}
    Tutkinto-ohjelma\\
   
\end{tabular}}
\\ \hline
\multicolumn{3}{|l|}{\begin{tabular}[c]{@{}l@{}}
    Työn ohjaaja(t)\\
\end{tabular}}
\\ \hline
\multicolumn{3}{|l|}{
\begin{tabular}[c]{@{}l@{}}
    Toimeksiantaja(t)\\
    
\end{tabular}}
\\ \hline
\multicolumn{3}{|l|}{
\begin{tabular}[c]{@{}l@{}}
Tiivistelmä  \\ \\

\end{tabular}}
\\ \hline
\multicolumn{3}{|l|}{\begin{tabular}[c]{@{}l@{}}
Avainsanat (subjects)\\

\end{tabular}}
\\ \hline
\multicolumn{3}{|l|}{\begin{tabular}[c]{@{}l@{}}
Muut tiedot \\ \\

\end{tabular}}
\\ \hline
\end{tabular}
\end{table}
\FloatBarrier
\clearpage

%\clearpage
\pagenumbering{arabic}
\tableofcontents
\newpage


\listoffigures
\listoftables


%\lstlistoflistings
\pagebreak

\section*{Without numbering}
LPI \quad Lorem IPsum

LPI \quad Lorem IPsum

\section{test document}

This is a test document.

Suspendisse consequat lectus urna, vel hendrerit tortor euismod quis. Aliquam
diam urna, rhoncus at suscipit nec, pellentesque sed tellus. Fusce auctor
dignissim purus vel maximus. Morbi faucibus, lectus eget tempor posuere, tellus
ex imperdiet nibh, sed ultricies risus diam vitae nisi. Sed sed augue
malesuada, fringilla sem rutrum, molestie erat.  Phasellus non fringilla ex.
Integer eu lacinia est. Mauris commodo arcu eu consectetur condimentum. Donec
at ipsum at nisl blandit commodo a vitae nulla.  Quisque luctus sit amet turpis
vitae auctor. Nunc blandit metus ligula, nec auctor tortor sollicitudin et.
Nullam tristique efficitur ipsum, vel laoreet sapien lobortis at. Ut aliquet,
nulla id lobortis scelerisque, neque ante hendrerit turpis, vitae maximus sem
neque quis felis. Ut metus ante, sodales eu tincidunt eu, lobortis id mauris.



\image[2in]{jamk.png}{fig:jamk}{Test caption}

Let's reference the test picture! Picture \ref{fig:dice} is a picture of
dice!


Suspendisse consequat lectus urna, vel hendrerit tortor euismod quis. Aliquam
diam urna, rhoncus at suscipit nec, pellentesque sed tellus. Fusce auctor
dignissim purus vel maximus. Morbi faucibus, lectus eget tempor posuere, tellus
ex imperdiet nibh, sed ultricies risus diam vitae nisi. Sed sed augue
malesuada, fringilla sem rutrum, molestie erat.  Phasellus non fringilla ex.
Integer eu lacinia est. Mauris commodo arcu eu consectetur condimentum. Donec
at ipsum at nisl blandit commodo a vitae nulla.  Quisque luctus sit amet turpis
vitae auctor. Nunc blandit metus ligula, nec auctor tortor sollicitudin et.

Time for a table!



Let's cite a reference without any additional information: \cite{singularity}

Let's cite a reference some additional information: \cite[p. 123]{singularity}

\jtable
    {Different types of dice}   % Caption
    {tbl:dicetypes}             % Label that is used to refence the table
    {l r l}                     % Table layout
    {
        % Table data
        \textbf{Type} & \textbf{Number of sides} & \textbf{Usage} \\
        D4 & 4 & Tabletop RPGs \\
        D6 & 6 & Gambling, games... \\
        D10 & 10 & Tabletop RPGs \\
        D20 & 20 & Tabletop RPGs \\
        D100 & 100 & Tabletop RPGs \\
    }
    
Suspendisse consequat lectus urna, vel hendrerit tortor euismod quis. Aliquam
diam urna, rhoncus at suscipit nec, pellentesque sed tellus. Fusce auctor
dignissim purus vel maximus. Morbi faucibus, lectus eget tempor posuere, tellus
ex imperdiet nibh, sed ultricies risus diam vitae nisi. Sed sed augue
malesuada, fringilla sem rutrum, molestie erat.  Phasellus non fringilla ex.
Nullam tristique efficitur ipsum, vel laoreet sapien lobortis at. Ut aliquet,
nulla id lobortis scelerisque, neque ante hendrerit turpis, vitae maximus sem
neque quis felis. Ut metus ante, sodales eu tincidunt eu, lobortis id mauris.

Let's reference our fine table: Table \ref{tbl:dicetypes} contains information
about different kinds of dice!


\subsection{lorem ipsum}

Suspendisse consequat lectus urna, vel hendrerit tortor euismod quis. Aliquam
diam urna, rhoncus at suscipit nec, pellentesque sed tellus. Fusce auctor
dignissim purus vel maximus. Morbi faucibus, lectus eget tempor posuere, tellus
ex imperdiet nibh, sed ultricies risus diam vitae nisi. Sed sed augue
malesuada, fringilla sem rutrum, molestie erat.  Phasellus non fringilla ex.
Nullam tristique efficitur ipsum, vel laoreet sapien lobortis at. Ut aliquet,
nulla id lobortis scelerisque, neque ante hendrerit turpis, vitae maximus sem
neque quis felis. Ut metus ante, sodales eu tincidunt eu, lobortis id mauris.

\begin{lstlisting}[language=python, caption=Example code block]
# Import the modules
import sys
import random

ans = True

while ans:
    question = raw_input("Ask the magic 8 ball a question: (press enter to quit) ")
    
    answers = random.randint(1,8)
    
    if question == "":
        sys.exit()
    elif answers == 1:
        print "It is certain"
    elif answers == 2:
        print "Outlook good"
    elif answers == 3:
        print "You may rely on it"
    elif answers == 4:
        print "Ask again later"
    elif answers == 5:
        print "Concentrate and ask again"
    elif answers == 6:
        print "Reply hazy, try again"
    elif answers == 7:
        print "My reply is no"
    elif answers == 8:
        print "My sources say no"
\end{lstlisting}

\subsubsection{more lorem ipsum}

Suspendisse consequat facilisis lacus, eget varius neque. Mauris blandit id
tellus vel consectetur. Integer porta tempor arcu, quis sodales urna posuere a.
Phasellus a lacinia dolor. Nam nec dui massa. Praesent vestibulum purus ac
felis volutpat vehicula. Sed sem nisl, hendrerit id gravida at, condimentum
hendrerit massa.  Maecenas vitae erat laoreet, semper enim sit amet,
condimentum ipsum. \cite{singularity2}

Donec at porttitor nibh. Suspendisse feugiat consequat ornare.  Mauris varius
porttitor libero ut facilisis. Pellentesque quis eros eros. Donec quis cursus
lorem. In eget diam felis. Sed dictum, tellus bibendum dictum commodo, ligula
felis semper nisl, ac molestie magna lacus vitae turpis. In volutpat nunc at
finibus vehicula. Vestibulum pretium at nibh in tempor. Cras sed mi sit amet
orci scelerisque mollis. Donec aliquet laoreet augue, ut malesuada massa semper
a. Suspendisse ac mi luctus, fringilla odio pellentesque, congue lorem.
Curabitur varius nunc eu elit mattis, sed gravida urna hendrerit. Duis eget
enim eget massa faucibus finibus. Suspendisse potenti. Interdum et malesuada
fames ac ante ipsum primis in faucibus.

\newpage
\renewcommand{\baselinestretch}{1}
\printbibliography
\addcontentsline{toc}{section}{\protect\numberline{}References}

\end{document}
